\documentclass[instructions]{uqthesis}
%\documentclass[final]{uqthesis} 


%*************************************
% FOR YOUR FINAL THESIS
%*************************************

%IMPORTANT! 
%The default document class (above - line 1 & 2) for the template is \documentclass[instructions]{uqthesis} - this document class will show instructional material and examples relevant to the preliminary material in the compiled PDF preview. THESE INSTRUCTIONS ARE FOR YOUR REFERENCE ONLY AND ARE NOT TO BE INCLUDED IN YOUR FINAL THESIS! 

%To turn off these instructions in your final thesis you MUST use the document class \documentclass[final]{uqthesis} 
%To activate the final thesis document class you must UN-COMMENT THIS DOCUMENT CLASS (remove the % from the start of line 2) and comment out the instructional document class on line 1 (add % to the start of line 1). 

%*************************************
% Introduction to template
%*************************************
%This is The University of Queensland Graduate School Official LaTeX Thesis template.

%Be sure to observe the content of comments within the source code, these are prefaced with a percentage symbol.
%Most important instructions have been CAPITALISED.
%To uncomment an inactive command (if required) remove the % from in front of the command.

%Please see the README for more information.

%This file loads the necessary packages, sets the page styles, and defines required macros.
%Edit this if you are comfortable with LaTeX.

%Other tweaks can be made in uqthesis.cls, but change these at your own risk!

%See README for version.

%You must have the memoir class installed.

\input{LaTexPackages.tex}


% ***************************************************
% Title page
% ***************************************************
%***THESIS TITLE***
\title{FPGA Web Interface and Ethernet Network Controller using the RISC-V Processor }
\subtitle{Project Proposal}

\author{Matthew Gilpin}
\studentnumber{45801600}


\currentdegrees{Project Proposal}

\date{Semester 1, 2023}
\submittedfor{Thesis - REIT4841}


\school{School of Information Technology and Electrical Engineering}

\begin{document}

\frontmatter
% Assemble title page
\maketitle


\clearpage
%YOU MUST EDIT THIS DOCUMENT.
% ***************************************************
% PRELIMINARY PAGES
% ***************************************************
% The instructions contained within this part of the thesis template need to be suppressed from the final thesis. There are instructions on how to do this in the MainThesis.tex file.

% To ensure your work is not suppressed with the instructions please add your text only where instructed.


\clearpage
\pagestyle{headings}


%***Table of Contents***
%These generate the table of contents, list of figures, and list of tables from items tagged with a \label{} command.
\tableofcontents
	\clearpage
\listoffigures
\listoftables

% \input{./PreliminaryAndBackPages/Symbols.tex} %List of symbols. REMOVE IF NOT NEEDED.

%***End of front matter***

% ***************************************************
% Thesis Content
%****************************************************
\mainmatter


%CHAPTER 1
\chapter[Introduction]{Introduction}
\label{Chap:Intro}

% ***************************************************
% Introduction
% ***************************************************



This chapter provides the necessary background and reasoning behind the proposed project. 

\section{Background }
In a technology age of growing numbers of cyber attacks and growing number of Internet-of-Things (IoT) devices being interconnected, it's 
paramount to ensure these devices operate safetly and securely. 

\section{Topic}

% ********* Enter your text below this line: ********

There are a plethora of different ways to reduce the likelyhood of cyber attacks.
A common approach is to employ a firewall to filter out potentially malicious packets. 

This project focuses primarily on securing edge IoT Ethernet networks. 


\section{Aims}

The aims of the proposed FPGA Ethernet controller and web interface on a RISC-V processor are:

\begin{itemize}
    \item Increase security to edge IoT networks.
    \item Increase the power efficiency for wire-speed firewalls.
\end{itemize}





%CHAPTER 2
\chapter[Literature review]{Literature review }
\label{Chap:label}	%CREATE YOUR OWN LABEL.
\pagestyle{headings}



Some of the concepts behind the proposed project, such as an Ethernet MAC or RISC-V processor are not new. Consequently, there is a variety of previous work 
in these areas. This part of the proposal will explore the prior work related to the project. 


\section{Field Programmable Gate Arrays}
\label{subsection:fpga}	
First introduced by Xilinx in 1984, field programmable gate arrays (FPGAs) allowed for large custom logic designs to be recognised without the need for 
expensive application specific integrated circuits (ASICs). More importantly, FPGAs did not suffer from the same scalability issues that
programmable array logic (PAL) encountered and has allowed for larger and more complex designs \cite{30YearsOfFPGA}. 

A big advantage to custom logic is the ability to create highly parallelised designs with lower latencies than software based serialised algorithms. This comes down to 
having a great degree of freedom when it comes to designing the architecture and ability to optimise for specific tasks.
As such, FPGAs have became ubiquitous in both digital signal processing and for accelerating an assortment of heterogeneous computing architectures and processes \cite{FPGAComputing}.
System on chip (SoC) design with custom hardware acceleration modules is an active area research. As \cite{FPGAComputing} points out, there is a focus towards 
using both hardware and software in \textit{edge} devices due to growing numbers of IoT devices.


Several papers, \cite{LwIPFPGAFirewall} \cite{IPFPGAFirewall2000} \cite{packetFilteringFPGA}, have proposed a range of other related FPGA based firewalls that have 
different properties and focus on different optimisations. The key benefit to these firewalls is their high performance - namely, low latency, and high throughput. 
Article \cite{LwIPFPGAFirewall} proposed an Ethernet firewall using LwIP (A TCP/IP stack) with five-tuple binding (the five filtered parameters in packet filters) 
to achieve a throughput of 950Mbps with a latency of 61.266us. A conference proceeding in 2000 \cite{IPFPGAFirewall2000} used a comparator unit to check the 
fields of the IP headers obtained a filtering rate of 500,000 packets per second. 


The enabling concept behind the above FPGA based firewalls is SoC design which involves integrating multiple components into a single package, or in this case a 
single FPGA. Often these will include small softcore microprocessors and some custom hardware such as the Ethernet or packet filtering like the proposed packet filters in \cite{LwIPFPGAFirewall}.
Having a microprocessor in the FPGA design can significantly reduce the complexity of the design and allows for quick and easy development in software instead of 
hardware \cite{SoftcoreBasedEmbeddedSystems}. In FPGA design, softcore processors are configurable and can be modelled in a hardware description language (HDL) 
which can then be synthesised onto ASICs or FPGAs hardware \cite{SoftcoreBasedEmbeddedSystems}. There are several softcore processors available for FPGA 
designs including ARM Cortex, Nios II, MicroBlaze, and RISC-V. 
 
While recently the royalty free RISC-V based cores have been popular amongst many SoC designs, other older processors are still common in the literature. The two 
big FPGA vendors, Xilinx (now AMD) and Altera (now Intel) have their own RISC based softcores. As an example, Janik et al. \cite{LwIPMicroblaze} used Xilinx's MicroBlaze processor 
as a media converter between optical (SFP interface) and copper (Ethernet) networks. Likewise, Altera's Nios II can be found in a variety of research papers 
including an embedded web server which significantly simplified the design \cite{NiosIIWebserver}. 



\section{Packet Filter Firewall}

Usually, the first line of defence against bad actors, firewalls play a vital component in computer networks and as such can become vastly complex. 
In essence, the job of a firewall is to isolate and restrict access to an internal network from an external one to increase security \cite{BuildingInternetFirewalls}.

There are several types of firewalls such as packet filters (PF), stateful packet firewalls and application firewalls \cite{FirewallsBook}. 
Traditional PFs are considered as stateless and filter exclusively on the fields in the network (layer 2) and transport 
(layer 3) layer headers \cite{FirewallsBook}. Such fields include IP addresses, port numbers and protocol type.

Due to this, PFs are inherently simple and efficient. Consequently, they are widely available and can be either implemented in software or in 
hardware \cite{BuildingInternetFirewalls}. The book, \cite{BuildingInternetFirewalls}, also highlights some inherent flaws with PFs which include not being able 
to suppress sophisticated attacks and in some cases, can be challenging to properly configure. More advanced firewalls can perform deep packet inspection which 
explore the contents of the higher layers to better evaluate a packets true intention \cite{FirewallsBook}. 

While firewalls such as \textit{iptables} in Linux are software based, hardware acceleration can vastly improve the performance of a packet filter. As stated in section 
\ref{subsection:fpga}, hardware acceleration allows for parallelised algorithms to be executed independently of a central processing unit (CPU). Wicaksana and Sasongko, 
\cite{FastRecongifFPGAFirewall}, proposed a packet classification engine as shown in figure \ref{fig:fast-fpga-classifier}. To obtain a fast and reconfigurable packet 
classifier, the authors of \cite{FastRecongifFPGAFirewall} used a hierarchical tree-based algorithm that inspects the multidimensional fields of the IP header through 
the use of parallel decision trees.

Essentially, the architecture in figure \ref{fig:fast-fpga-classifier} employs memory to store the ruleset and uses a multiplexer and a comparator to evaluate each of the fields 
in the header. As an safegaurd, the authors opted for a \textit{default-deny} ruleset to prevent any unwanted traffic. 


\begin{figure}[h]
    \centering
    \includegraphics[width=0.8\textwidth]{Images/packetFilterHardware.png}
    \caption[Packet classifier]{Packet classifier \cite{FastRecongifFPGAFirewall}}
    \label{fig:fast-fpga-classifier}
\end{figure}


\newpage

Wasti \cite{Wasti2001HardwareAP} presents several other classification algorithms for both hardware and software packet filters. \textit{'Sequential matching'} provides the most 
trivial solution as it matches each rule to the incoming packet. While simple, this design has scalability issues as more rules get added. Another method proposed in 
\cite{Wasti2001HardwareAP} is by using a \textit{'Grid of tries'} which uses tries (a type of tree datastructure) to help pattern match the packets, but fails to extend to multiple fields. 
Hardware algorithms using \textit{Ternary CAMs} (stores words with 3-valued-digits - namely '0', '1' and '*') and \textit{Bit-parallelism} were also discussed. Both of these 
exploited the parallelised nature of hardware design. One limiting factor with the classification methods cited in \cite{Wasti2001HardwareAP} is their configurability and 
expandability. 



\section{RISC-V processor}
In the world of processor architectures, there are four major families, namely AMD64, x86, ARM and RISC-V. The two former instruction set architectures (ISA) 
are apart of the complex instructions sets (CISC) and are found in the majority of computers. ARM and RISC-V have a reduced instruction set compared to the CISC family and 
subsequently fall under the RISC family and are ideal for low power microprocessors \cite{RV16Embedded}.

RISC-V is an open and royalty free ISA and as a result, a plethora of softcore based custom implementations have been designed \cite{CatalogRISCSoftcore}. 
Consequently, there is an abundance of articles delving into RISC-V from evaluating the ISA \cite{InvestigatingRiscv} to creating multicore architectures
\cite{RiscVMulticore}. A 2019 paper, \cite{CatalogRISCSoftcore} evaluated a variety of different RISC-V softcore processors. RISC-V International have 
also published a list\footnote[1]{See: https://github.com/riscv/riscv-isa-manual/blob/master/marchid.md} of different RISC-V implementations 
that have a unique architecture ID. The majority of these are either written in a HDL for either application specific integrated circuits (ASICs) or FPGAs.
The \textit{NEORV32 RISC-V} softcore processor is written purely in vendor-agnostic VHDL and importantly has a considerable amount of documentation. 

Being a softcore processor, control is given over which modules are implemented. Some basic features of the \textit{NEORV32 RISC-V} include 
UART, SPI, and GPIO interfaces \cite{neorv32Datasheet}. The datasheet, \cite{neorv32Datasheet}, also mentions that it supports a \textit{'Wishbone b4 classic'} 
external bus interface. A Wishbone B4 (or just 'wishbone') interconnection is designed specifically to connect modular pieces of hardware together on a 
SoC into the memory mapped 32bit address space in the processor \cite{WishboneSpec}. This approach has the benefit of not needing to create custom 
instructions for the microprocessor. 


\section{Ethernet MAC}

First introduced in 1983, the IEEE 802.3 standard \cite{IEEE802.3-2012}, more commonly known by the name of 'Ethernet', defines the \textit{'Medium Access Control'} 
(MAC) protocol amongst other things for two or more devices to communicate over a network. This standard is just one part in the layered network 
models such as the OSI model or TCP/IP model, namely the network layer - layer 2. 


A core function of the Ethernet MAC is to attach the required MAC layer headers to the head and tail of the layer 3 payload to create an Ethernet packet. The fields 
in an Ethernet packet can be seen in figure \ref{fig:ieee-mac-headers}. 

\begin{figure}[h]
    \centering
    \includegraphics[width=0.65\textwidth]{Images/mac_packet.png}
    \caption[MAC layer headers]{MAC layer headers \cite{IEEE802.3-2012}}
    \label{fig:ieee-mac-headers}
\end{figure}

After the packet has been constructed, the data is forwarded out to the physical (PHY) layer 
least significant bit (LSB) first \cite{IEEE802.3-2012}. Typically, a PHY management chip is used to handle the physical layer channel encoding amongst other things. 
These PHY chips often can be interfaced with the media independent interfaces such as MII, RMII, GMII and RGMII \cite{OptimisedEthernetMAC}. The reduced media 
independent interface (RMII) is one of these standards defined in \cite{IEEE802.3-2012} and consists of a reference clock, 2 bit wide transmit (TX), 2 bit wide 
receive (RX) lines and a few other supplementary signals as defined in the LAN8720A datasheet \cite{LAN8720ADatasheet}.


The MAC layer itself is usually implemented in hardware as it has several advantages over a software implementation. The core reasons behind this are due to parallelised nature of FPGAs and that parts of the MAC can operate independently \cite{reducedEtherentMacFPGA}. One key example is the calculation of the 
frame check sequence (FCS in figure \ref{fig:ieee-mac-headers}). The FCS for Ethernet is a 32bit cyclic redundancy check (CRC) \cite{IEEE802.3-2012} and 
in addition to Etherent, the CRC32 can be found in an extensive amount of applications. As such, research has been conducted into parallelising the calculation. 
Noteably, Mitra and Nayak \cite{ParallelCRC} proposed a low latency parallelised architecture for FPGA design on CRC32. As a result, packets can be assembled 
faster and offload additional processing burden from the CPU. 


Numerous articles \cite{OptimisedEthernetMAC} \cite{EthernetAXI} \cite{EthernetRMII} can be found about Ethernet MACs implemented 
on FPGAs each with a slightly different approach. Fundamentally though, as best highlighted in \cite{OptimisedEthernetMAC}, a simple way of implementing a MAC is by employing a finite state 
machine (FSM) to set the required fields. Another technique found in these articles is the use first-in first-out (FIFO) buffers to cross clock domains. This is a common technique used 
in FPGA design as it allows you to have the packet assembly logic at a much higher clock rate than the output RMII reference clock speed \cite{EthernetAXI}. 

In addition to the papers, there are a plethora of intellectual property (IP) blocks for xMII interfaces in HDL 
which have their own benefits and drawbacks. Some freely available HDL modules for Ethernet MACs can be found in both a complete \footnote[1]{See: https://github.com/yol/ethernet\_mac} \footnote[2]{See: https://github.com/alexforencich/verilog-ethernet/} 
\footnote[3]{See: https://opencores.org/projects/ethernet\_tri\_mode} and incomplete state
\footnote[4]{See: https://github.com/pabennett/ethernet\_mac}.






\section{Web servers and network stacks}

Almost all firewalls need to be configured with a ruleset which can be configured in two common ways, using a command line interface (CLI) 
or by a web-based graphical user interface (GUI). Before a web server can be realised, the network stack (Layers 3, and 4) need to be established since a web server 
operates at the application layer (layer 4). As embedded platforms are resource limited, special precautions need to be taken into consideration when it comes to memory and resource 
usage \cite{OptimCortexLwIP}.

Article \cite{LwIPFPGAFirewall} investigated using the open source lightweight IP (LwIP) network stack as a mechanism for interfacing with the firewall. 
The LwIP library is a popular lightweight TCP/IP stack which has been investigated in a plethora of research papers and projects \cite{ImprovemntOptimLWIP} 
\cite{OptimCortexLwIP}. Often these papers run LwIP on real time operating systems (RTOS) such as FreeRTOS or Zephyr.

FreeRTOS is a leading RTOS for microprocessors and is distributed freely under the MIT license. As an RTOS, it provides an abstraction to the hardware that allows 
for multitasking and brings other OS-Like features to embedded systems. Several ports are available including one for RISC-V. 

FreeRTOS also provide their own TCP/IP network stack called \textit{FreeRTOS-Plus-TCP} which includes a HTTP web server example and is much newer than LwIP.
Consequently, less research can be found apart from existing documentation. The library aims to provide a threadsafe Berkley sockets API and network stack 
supporting multiple protocols such as DHCP, DNS, TCP, and UDP \cite{FreeRTOSTCP}. LwIP is not threadsafe and typically suffers from memory issues as found 
in \cite{OptimCortexLwIP}.




%CHAPTER 3

% ***************************************************
% Example of an internal chapter
% ***************************************************
%This is an internal chapter of the thesis.
%If you have a long title, you can supply an abbreviated version to print in the Table of Contents using the optional argument to the \chapter command.
\chapter[Timeline and Plan]{Timeline and Plan}
\label{Chap:label}	%CREATE YOUR OWN LABEL.
\pagestyle{headings}



% ********* Enter your text below this line: ********
This section of the report details the plan and timeline of the proposed project. It also details the necessary risk assessment.
% ***************************************************

\section{Performance Indicators}



\section{Required Equiptment}
While the hardware design will be developed in such a way that it is vendor agnostic, to test the design a Digilent Nexys A7-100T developement board will be used.
Importantly, this board has a RJ45 connector and LAN8720 RMII interface chip which allows for a regular fast ethernet connection to be directly connected 
to the FPGA board. An additional LAN8720 ETH board from Waveshare is also required to obtain the secondary interface. 

To validate the functionality and effectiveness of the design, it will be compared with a Raspberry Pi Compute Module 4 (CM4) with a Waveshare CM4-DUAL-ETH-MINI 
daughterboard which contains two 1GbE interfaces. This will act as a baseline. 

\section{Technology Readiness Level}



\section{Milestones}
\label{Sec:label}	%CREATE YOUR OWN LABEL.

% ********* Enter your text below this line: ********
Table ~\ref{table:milestones} shows the tasks and expected durations of the proposed project.
  

\begin{table}[hbt!]
\centering%
% \begin{tabularx}{lll}
    \begin{tabularx}{\textwidth}{ lXl }
        \hline
        Task                            & Details & Duration  \\ \hline
        Create MAC                      &  Create custom Layer 2 Ethernet hardware based on the IEEE 802.3 standard      & 3-4 Weeks \\
        Wishbone Interface              &  Connect the Ethernet MAC to the NEORV32 RISC-V Processor using the wishbone interface and access it via software       & 2 Weels   \\
        Webserver                       &  Create and Get the webserver working on the NEORV32 Processor. Web page should be accessed from another computer       & 5-6 Weeks \\ 
        Firewall Hardware               &  Create the hardware between 2 Ethernet MACs to filter out packets based on rules      & 3-4 Weeks \\ 
        Integration with software       &  Add functionality to the server to be able to configure the firewall rules       & 1 Week \\ 
        Measure and Compare             &  Compare to pre-existing solutions       & 1 Week \\ 
        \hline
        \end{tabularx}
\caption{ Milestones for the proposed project}\label{table:milestones}
\end{table}
 

\section{Project Risk Assessment}

The majority of the work compeleted in the proposed project is digital and poses little risk outside of the standard office sitting. 

\begin{table}[hbt!]
    \centering%
    % \begin{tabularx}{lll}
        \begin{tabularx}{\textwidth}{ lllX }
            \hline
            Risk       & Severity & Likelyhood & Mitigation  \\ \hline

            Licensing & Minor & Moderate & Avoid software/hardware that requires a specific license. \\
            Data loss & Catastrophic & Unlikely & Ensure all items are backed-up to the cloud and use services such as GitHub where appropriate. Employ a 3 2 1 backup strategy \\
            Hardware Failure & Moderate & Unlikely & Double check all connections to the FPGA board before powering. Reduce excessive handling where necessary to minimise risk of damaging the equiptment \\
            % Electrocution from PoE Ethernet & High & Low & Ensure all items are backed-up to the cloud and use services such as GitHub where appropriate. \\
            Illness & High & Likely & Take breaks periodically to avoid being overworked, and take necessary recovery steps if sick. \\
            Missed Deadlines & Major & Likely & Ensure plans are followed and compelete tasks as soon as possible. If behind, spend extra time on project to catch up.\\
    
            \hline
            \end{tabularx}
    \caption{ Risk assessment of proposed project}\label{table:milestones}
    \end{table}
     
    

% HOW TO ADD ADDITIONAL CHAPTERS
% Step One: Add a new folder called "ChapterX" (X being the chapter number).
% Step Two: Within the folder add a new .tex file by clicking the "New File" button in the Overleaf Menu. Rename the file to a title of your choice.
% Step Three: Copy the Chapter 2 headline and "\input" command located above and insert it below Chapter 2.
% Step Four: Rename the headline to your specific chapter number, change the input command to include the name of the folder you created and the name of the file you created.
% Repeat this process for every chapter.

%CONCLUSION CHAPTER
\chapter[Conclusion]{Conclusion}
\label{Chap:Conclusion}

\section{Summary}

This thesis explored the design and implementation of a hardware packet filter, Ethernet MAC with a RISC-V softcore processor. The Ethernet MAC and packet filter were created from scratch, while the NEORV32 RISC-V softcore was used to interface with the custom hardware. This was all implemented on a Xilinx Artix 7 FPGA board with the LAN8720A PHY. The design was evaluated and compared against similar preexisting solutions on the market. While the design in this thesis did not outperform the preexisting solutions in all cases, it was comparable and did provide new and unique features not seen before in the embedded systems space. In addition to the hardware, a webserver and web application was created to allow for easy configuration of the packet filter. 

The design in this thesis shows that hardware packet filters in embedded systems are feasible and consume minimal resources while providing great performance. The design was able to achieve a latency of $4\mu s$ while only consuming 571 slice LUTs and 1145 slice registers. The packet filter design was also able to achieve a power consumption of just 2mW.



\section{Limitations}
While this thesis explored the design and implementation of a hardware packet filter, Ethernet MAC with a RISC-V softcore processor, there are some limitations to the design and the research conducted. These can be summarised in the following points:


\begin{itemize}
    \item The current system only supports filtering network packets in one direction as it assumes all packets leaving the device is safe. 
    \item The design only considers IPv4 packets without IEEE 802.1Q VLAN tagging.
    \item The transmit logic is not optimised for resource usage and can be improved.
    \item Only one interface is supported due to the bandwidth limitations of the PMOD ports on the Nexys A7 board.
    \item Only HTTP and other unencrypted protocols are supported.
    \item The current design is bottlenecked from the processor - NEORV32
\end{itemize}


In addition, only one interface is supported due to the Nexys A7 board only consisting of one ethernet PHY and the additional PMOD ports are not suitable for Ethernet. This is because the PMOD ports are only rated for a 25Mhz bandwidth, while the RMII signals are 50Mhz. As such, signal integrity issues arose (see appendix \ref{app:eye_diagrams} for eye diagram) and restricted the use to just one interface - the onboard PHY. A new development board with two PHYs would be needed.











\section{Sustainability}



The sustainability of the designed system in this thesis is multi-layered with considerations in hardware, software, and web development taken into consideration. 

Starting at the hardware level, the specific implementation in this thesis uses the Digilent Nexys A7 FPGA board with a Xilinx Artix 7 FPGA and a LAN8720A PHY. The LAN8720A PHY uses the standardised RMII interface, increasing the portability across various FPGAs assuming adequate resources. Likewise, other RMII PHY chips could be swapped out with the LAN8720A without issue due to the standardised interfae. The specific implementation of the hardware in this thesis requires minimal modification to support other media independent interfaces such as RGMII or XGMII. The main difference would be in the input and output FIFOs being able to support the different clock rates and bit widths. Apart from the clocking IP block, the design is written in vendor-agnostic VHDL and can be easily ported to other FPGAs. 

The TCP/IP standards including the IEEE 802.3 Ethernet standards and protocols such as IP, TCP, and UDP, have being around for decades with minimal adjustments to the standards since their inception. New features in these standards typically are additive and do not modify the packet structure itself, allowing for backwards compatibility. An example of this is the introduction of IPv6 in 1998 \cite{rfc2460}. This is important when designing a hardware layer packet filter which assumes the bit positioning of the fields in the packet. Considering this, the packet filter should be still applicable in the future as the packet structure is unlikely to change.


The choice of using a RISC-V processor architecture is another sustainable consideration made in this project. While RISC-V is royalty-free and the core instruction set architecture is open-source, not all implementations of RISC-V cores are open-source or free. The specific implementation of RISC-V used in this thesis is the NEORV32, which is open-source and free to use under the BSD-3-Clause license \footnote[1]{See: https://github.com/stnolting/neorv32/blob/main/LICENSE}. This importantly allows for commercial use and redistribution, but does not carry any liability. Additionally, the NEORV32 is still in active development and is continually getting updated with new features and bug fixes. While this means that new features and security patches will get added, it also incurs additional work to update the design to the latest version. The future of the open-source design is also vulnerable to becoming abandoned if the developer decides to stop working on it. 


Similarly to the NEORV32, FreeRTOS and their first-party FreeRTOS-Plus-TCP library are also open-source and free to use under the MIT license \footnote[2]{See: https://www.freertos.org/a00114.html}. Like the NEORV32, FreeRTOS and it's libraries are actively maintained by Amazon and feature updates, albeit less frequently than the NEORV32. The FreeRTOS-Plus-TCP library is also feature-rich and is continually getting updated, however, it's documentation and community support is more limited than that of LwIP.



In terms of web development, Vue.js was used as a framework for web development. Primary issues of concern for web applications are dependencies and library support and maintenance. In this project, only a small handful of packages were used, each of which are well maintained and have a large community support. If future designs were to use Vue.js with other packages, it is recommended that the packages used are well maintained. Like the TCP/IP stack, HTTP, HTML, Javascript and CSS, which Vue.js is built upon, are heavily standardised but do change over time. As such, future designs should be aware of these changes and adapt accordingly.


Finally, the security of the design poses the greatest risk to sustainability. Malicious bad actors are continually innovating and finding new ways to breach systems. Despite this, the core design of the packet filter remains fundamentally strong due to it's basic filtering capabilities. While it will not prevent all cyber attacks, it can be used as a tool in a much larger system to help mitigate the risk of such attacks. Further fortifications can be made to the system architecture by using HTTPs, public key cryptography, and deep packet inspection.





\section{Recommendations and future work}

In light of the findings in this thesis, several key recommendations can be made for future work in the area of embedded system SoC design which features Ethernet connectivity.

The primary recommendation is to incorporate dedicated hardware packet filtering into SoC designs. As demonstrated in this thesis, the resource utilisation for the packet filtering logic is minimal (571 slice LUTs and 1145 slice registers) and can be easily integrated into the design with minimal impact to cost. Superior latency, throughput and power consumption metrics are only some of the benefits presented in this thesis over the conventional software based packet filters. Additionally, the potential resilience against potential security vulnerabilities is another key advantage of this approach.

While the NEORV32 is a solid general-purpose softcore processor, it seemed to bottleneck the design and witheld the design from achieving better performance. The CVITEK CV1800B, used in the MilkV-Duo and compared in this thesis, is a powerful SoC with an abundance of resources including a hardware MAC and PHY, but falls short of including a hardware packet filter. An ideal choice would be to have the performance of the CVITEK CV1800B with the hardware packet filtering capabilities of the design presented in this thesis. This would give the best of both worlds, performance and security.

Alternatively, research into using hybrid SoC FPGAs such as the Xilinx Zynq lineup which include an FPGA and a hardcore processor connected over a high speed fabric could be a good avenue. This provides the flexibility of an FPGA with the performance of a hardcore processor, ideal for small scale designs that would otherwise be too expensive for custom silicon.

Leveraging single page application frameworks such as Vue.js for use in embedded systems is another recommendation resulting from the work done in this thesis. Light-weight applications and low power devices can benefit greatly from the use of such frameworks as fewer network traffic is required due to client-side routing and static web content. In combination with a lightweight API, dynamic data can be obtained with minimal network traffic, making the user experience seamless and responsive.

In addition to these recommendations for future designs, the work in this thesis can be extended in the following areas:

\begin{itemize}
    \item Redesign of the transmit logic to consume less resources while not losing on speed/performance, 
    \item Add a second Ethernet interface to filter traffic for other devices on the network, 
    \item Utilise the DDR2 RAM to free up BRAM in the FPGA, 
    \item Implement public key cryptography for HTTPS,
    \item Support faster media interfaces, eg RGMII for 1Gbit/s or XGMII for 10Gbit/s,
    \item Use a different bus interconnect for the processor, eg AXI4, and
    \item Look into using LwIP over FreeRTOS-Plus-TCP.
\end{itemize}






% ***************************************************
% Bibliography
%****************************************************
\bibliographystyle{ieeetr}

\bibliography{./References/Bibliography}

%When you have finished your thesis we recommend that you manually fix any errors in your bibliography. 
%To do this, compile, copy the .bbl into a new .tex file and include this here after commenting out the other bibliography commands. Make corrections in that .tex file.

% ***************************************************
% Appendices
%**************************************************** 
%UNCOMMENT THIS SECTION IF YOU ARE USING APPENDICES.
%Simply adapt the same formatting used for other chapters.
\appendix
% If you need appendix in your thesis then consider the following appendix file (you can add more if you need more) otherwise you should not consider it in your main thesis.
% ***************************************************
% Appendix
% ***************************************************
\chapter{Appendix}

Write your appendix here. Following two are examples. 


\section{Neorv32 memory address space layout}
\label{app:mem_address}
\begin{figure}[h!]
    \centering
    \includegraphics[width=1\textwidth]{Images/neorv32_address_space.png}
    \caption{Neorv32 Memory Address space.}
\end{figure}



\section{FPGA primitives utilisation}
\label{app:res_usage}
\begin{table}
    \centering
    \caption{FPGA primitives utilisation for XC7A100T}
    \begin{tabular}{|l|r|l|}
        \toprule
        Ref Name   & Used & Functional Category \\
        \midrule
        LUT6       & 16262 & LUT \\
        LUT5       & 14820 & LUT \\
        FDRE       & 14500 & Flop \& Latch \\
        LUT3       & 13222 & LUT \\
        MUXF7      &  2436 & MuxFx \\
        FDCE       &  1875 & Flop \& Latch \\
        RAMD64E    &  1836 & Distributed Memory \\
        LUT4       &  1294 & LUT \\
        LUT2       &  1016 & LUT \\
        MUXF8      &   884 & MuxFx \\
        CARRY4     &   437 & CarryLogic \\
        LUT1       &   156 & LUT \\
        RAMB36E1   &   130 & Block Memory \\
        FDPE       &    41 & Flop \& Latch \\
        OBUF       &    40 & IO \\
        LDCE       &    36 & Flop \& Latch \\
        IBUF       &    24 & IO \\
        SRLC32E    &    21 & Distributed Memory \\
        OBUFT      &    11 & IO \\
        BUFG       &     8 & Clock \\
        FDSE       &     5 & Flop \& Latch \\
        DSP48E1    &     4 & Block Arithmetic \\
        SRL16E     &     1 & Distributed Memory \\
        MMCME2\_ADV &    1 & Clock \\
        \bottomrule
    \end{tabular}
\end{table}

\begin{table}
    \centering
    \caption{Memory Utilisation}
    \begin{tabular}{|l|r|r|r|r|r|}
        \toprule
        Site Type      & Used & Fixed & Prohibited & Available & Util\% \\
        \midrule
        Block RAM Tile &  130 &     0 &          0 &       135 & 96.30 \\
        RAMB36/FIFO*   &  130 &     0 &          0 &       135 & 96.30 \\
        RAMB36E1 only  &  130 &     - &          - &         - &    -  \\
        RAMB18         &    0 &     0 &          0 &       270 &  0.00 \\
        \bottomrule
    \end{tabular}
\end{table}

\begin{table}
    \centering
    \caption{Slice Logic Utilisation}
    \begin{tabular}{|l|r|r|r|r|r|}
        \toprule
        Site Type                 & Used & Fixed & Prohibited & Available & Util\% \\
        \midrule
        Slice LUTs*               & 40920 &     0 &          0 &     63400 & 64.54 \\
        LUT as Logic              & 39062 &     0 &          0 &     63400 & 61.61 \\
        LUT as Memory             &  1858 &     0 &          0 &     19000 &  9.78 \\
        LUT as Distributed RAM    &  1836 &     - &          - &         - &    -  \\
        LUT as Shift Register     &    22 &     - &          - &         - &    -  \\
        Slice Registers           & 16457 &     0 &          0 &    126800 & 12.98 \\
        Register as Flip Flop     & 16421 &     0 &          0 &    126800 & 12.95 \\
        Register as Latch         &    36 &     0 &          0 &    126800 &  0.03 \\
        F7 Muxes                  &  2436 &     0 &          0 &     31700 &  7.68 \\
        F8 Muxes                  &   884 &     0 &          0 &     15850 &  5.58 \\
        \bottomrule
    \end{tabular}
\end{table}


\section{Additional webpages built into the webserver.}
\label{app:additional_webpages}
\begin{figure}[h!]
    \centering
    \includegraphics[width=1\textwidth]{Images/webapp_about.png}
    \caption{Screenshot of the about page in the webapp.}
    \label{fig:web_app_about}
\end{figure}

\begin{figure}[h!]
    \centering
    \includegraphics[width=1\textwidth]{Images/webapp_config.png}
    \caption{Screenshot of the config page in the webapp.}
    \label{fig:web_app_config}
\end{figure}


\section{UDP ping times between boards}
\label{app:udp_ping_measurements}
\begin{table}[ht]
    \centering
    \caption{Average UDP RTT for different devices and payload sizes.}
    \begin{tabular}{lcc}
    \toprule
    Device & 7 bytes (ms) & 256 bytes (ms) \\
    \midrule
    WIZ5500 Pico & 1.88 & 2.07 \\
    F767ZI & 1.30 & 1.39 \\
    F767ZI 80Mhz & 1.41 & 1.51 \\
    MilkV & 1.04 & 1.09 \\
    FPGA board & 1.45 & 1.72 \\
    \bottomrule
    \end{tabular}
    \end{table}

\section{Current measurments from boards}
\label{app:current_measurements}

Table \ref{tab:power_consumption} shows the current measurments. Notably Diff1 is the difference between the Idle and No ETH fields while Diff2 is the difference between the Idle and Clean states.

\begin{table}[ht]
    \centering
    \caption{Device power consumption data (all values in mA)}
    \label{tab:power_consumption}
    \begin{tabular}{lccccccc}
    \toprule
    Device    & Idle  & Busy  & Average  & No ETH  & Clean  & Diff1  & Diff2 \\
    \midrule
    FPGA      & 301.4     & 300.13    & 300.765      & 264         & 202.13           & 37.4     & 99.27       \\
    MilkV     & 76.27     & 76.53     & 76.4         & 69.23       & 69.23            & 7.04     & 7.04       \\
    F767ZI    & 209.25    & 206.9     & 208.075      & 153.46      & 121.31           & 55.79    & 87.94       \\
    WIZ5500   & 155.88    & 158.67    & 157.275      & 97.9        & 80.35            & 57.98    & 75.53       \\
    \bottomrule
    \end{tabular}
\end{table}


\section{Thermal measurements for boards}

\begin{table}[ht]
    \centering
    \caption{Device measurements over time using FLIR One thermal camera}
    \label{tab:measurements}
    \begin{tabular}{lcccccc}
    \toprule
    Device & 5min & 10min & 30min & 1h & 2h \\
    \midrule
    FPGA & 38 & 38.2 & 38.9 & 39.1 & 40.4 \\
    MilkV & 36.7 & 39.8 & 35.5 & 39.1 & 38.1 \\
    F767ZI (STM) & 35.9 & 38.9 & 35.8 & 37.5 & 36.8 \\
    F767ZI (PHY) & 38 & 38.8 & 35.2 & 37.2 & 36.2 \\
    WIZ5500 (RP2040) & 46.6 & 53.1 & 54.6 & 54.5 & 53.2 \\
    WIZ5500 (PHY) & 58 & 59 & 58.4 & 58.5 & 56.8 \\
    \bottomrule
\end{tabular}
\end{table}




\end{document}
