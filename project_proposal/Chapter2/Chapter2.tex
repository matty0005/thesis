
% ***************************************************
% Example of an internal chapter
% ***************************************************
%This is an internal chapter of the thesis.
%If you have a long title, you can supply an abbreviated version to print in the Table of Contents using the optional argument to the \chapter command.
\chapter[Abbreviated title]{Background}
\label{Chap:label}	%CREATE YOUR OWN LABEL.
\pagestyle{headings}



% ********* Enter your text below this line: ********
Introduce the broad layout of the chapter.

% ***************************************************


\section{Introduction}
\label{Sec:label}	%CREATE YOUR OWN LABEL.



PREVIOUS WORK

\section{Custom MAC}
The decision in creating a custom MAC might be considered as insteresting given the range of pre-existing Intellectual 
Property (IP) cores for Ethernet on FPGAs. The issue with the pre-existing solutions only have a single output to connect to something like a softcore
processor. To create a firewall, the network traffic would need to pass through the processor. To decrease latency, a second interface can be added 
to the MAC to allow traffic to flow through a hardware-based firewall. This is analogous to the direct memory access (DMA) controller on most modern 
microprocessors. 



\section{Packet Filter Firewall}

Usually, the first line of defence against bad actors, it is a vital component in a computer network and can become vastly complex. There are several
types of Firewalls such as packet filters (pf), stateful packet firewalls and application firewalls \cite{FirewallsBook}. Firewalls can also perform 
other tasks and employ other techniques to secure a network, however, in this project the most basic pf-style firewall will be implemented. 
Packet filters are considered as stateless and traditionally only filter on the fields in the headers in the network (layer 2) and transport 
(layer 3) layers \cite{FirewallsBook}. Such fields include IP addresses, port numbers and protocol type.

More advanced firewalls can perform deep packet inspection and explore the contents of the higher layers to better evaluate a packets true intention. 
While there is provision to add this functionality on an FPGA based firewall, this will not be explored in this project due to its significant increase 
in complexity. 


\section{RISC-V processor}






The IEEE 802.3 standard\cite{IEEE802.3-2012}, more commonly known by the name of 'Ethernet' defines the \textit{'Medium Access Control'} (MAC) 
protocol amongst other things for two or more devices to communicate over a network. This standard is just one part in the layered network 
models such as the OSI model and TCP/IP models. 


RISC-V, a reduced instruction set computer architecture, is an open and royalty free instruction set architecture (ISA). As a result, a plethora 
of soft-core processors have been made. The specific core proposed in this project is the 'NEORV32` RISC-V processor. It's a highly configurable 
microcontroller like system on chip (SoC) written purely in VHDL.

