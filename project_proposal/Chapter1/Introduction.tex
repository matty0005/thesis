\chapter[Introduction]{Introduction}
\label{Chap:Intro}

% ***************************************************
% Introduction
% ***************************************************



This chapter provides the necessary background and reasoning behind the proposed project. 

\section{Background }


In a technology age of growing numbers of cyber attacks and record number of connected devices, it's 
paramount to ensure these devices operate safely and securely. The Australian Cyber Security Center (ACSC) received in 
excess of 76,000 cybercrime reports and growing in the 2021-22 financial year \cite{acsc_2022}. The growing trend of Internet of Things (IoT) will provide 
more opportunity for black hats (malicious attackers). IHS Markit estimates 125 billion IoT devices will be connected by 2030 \cite{IHS_iot}. 

To cope with the increase in IoT devices, a common shift to edge computing has evolved in favour over the traditionally more centralised cloud computing 
architecture. The core principals behind the paradigm is to move the data processing closer geographically to its origin to not only decrease the 
central load, but to improve latency \cite{EdgeComputingPerspectives}. Due to the distributed load, smaller and more efficient computers can be used 
at the edge/perimeter of these networks \cite{EdgeComputingPerspectives}. Just like any other computer connected to the broader network, these 
edge networks also need to be protected from bad actors.




Edge computing as \cite{EdgeComputing} puts it, the paradigm envolves the computation and analysis of data 
at the \textit{edge} of the network to be as close as possible to the source of the data. This has many advantages including: lower latency, bandwidth requirements,
availability, energy, security and privacy \cite{EdgeComputing}.


