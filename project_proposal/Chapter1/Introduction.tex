\chapter[Introduction]{Introduction}
\label{Chap:Intro}

% ***************************************************
% Introduction
% ***************************************************



This chapter provides the necessary background and reasoning behind the proposed project. 

\section{Background }


In a technology age of growing numbers of cyber attacks and record number of connected devices, it's 
paramount to ensure these devices operate safetly and securely. The Australian Cyber Security Center (ACSC) recieved in 
excess of 76,000 cybercrime reports and growing in the 2021-22 financial year \cite{acsc_2022}. The growing trend of Internet of Things (IoT) will provide 
more opportunity for black hats (malicious attackers). IHS Markit estimates 125 billion IoT devices will be connected by 2030 \cite{IHS_iot}. 

To cope with the increase in IoT devices, a common shift to edge computing has evolved in favour over the traditionally more centralised cloud computing 
architecture. The core principals behind the paradigm is to move the data processing closer geographically to its origin to not only decrease the 
central load, but to improve latency \cite{EdgeComputingPerspectives}. Due to the distrbutied load, smaller and more efficient computers can be used 
at the edge/permimeter of these networks \cite{EdgeComputingPerspectives}. Just like any other computer connected to the broader network, these 
edge networks also need to be protected from bad actors.



\section{Topic}

While there is no single solution that will fully protect an edge network, a common and effective way to reduce the unauthorised/unwanted network 
traffic is by simply filtering out the potentially malicious packets. While this may seem overly complex, in reality a few simple rules can be followed
to decide on whether to forward or deny/drop/block packets from entering or exiting a network. These packet filters (pf) are a type of firewall that do 
not follow any complex rules and keep state between packets or use deep packet inpsection to check the contents of the payload to ensure it's not 
malicious. 

Packet filters are considered as stateless and traditionally only filter based on the fields in the headers at the network (layer 2) and transport 
(layer 3) layers \cite{FirewallsBook}. Such fields include IP addresses, port numbers and protocol type.

The proposed project consists of making a hardware implementation of a pf with custom Ethernet Media Access Controllers (MAC) connected to a hardware
based filtering block which is all controlled by a RISC-V softcore processor. This will then also have a web interface so that a user can configure
the rules for the pf.



\section{Aims}

The aims of the proposed FPGA Ethernet controller and web interface on a RISC-V processor are:

\begin{itemize}
    \item Increase security to edge IoT networks.
    \item Increase the power efficiency for wire-speed firewalls.
\end{itemize}



\section{Establishing Exclusions}

While the proposed project will reduce the likelyhood of network based attacks it is not a \textit{'one size fits all'} solution. 
By the nature of the IoT and edge network ecosystem, there are a myriad of different attack vectors where not all of them will be detectable at the
netowrk level.

The proposed project will \textbf{not}
\begin{itemize}
    \item Protect against all attacks
    \item Be able to protect against all IoT devices.
    \item Not perform routing
\end{itemize}