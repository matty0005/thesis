\chapter[Introduction]{Introduction}
\label{Chap:Intro}

% ***************************************************
% Introduction
% ***************************************************



This chapter provides the necessary background and reasoning behind the proposed project. 
\section{Topic}

% ********* Enter your text below this line: ********

In a technology age of growing numbers of cyber attacks and growing number of Internet-of-Things (IoT) devices being interconnected, it's 
paramount to ensure these devices operate safetly and securely. There are a plethora of different ways to reduce the likelyhood of cyber attacks.
A common approach is to employ a firewall to filter out potentially malicious packets. 

This project focuses primarily on securing edge IoT Ethernet networks. 


\section{Custom MAC}
The decision in creating a custom MAC might be considered as insteresting given the range of pre-existing Intellectual 
Property (IP) cores for Ethernet on FPGAs. The issue with the pre-existing solutions only have a single output to connect to something like a softcore
processor. To create a firewall, the network traffic would need to pass through the processor. To decrease latency, a second interface can be added 
to the MAC to allow traffic to flow through a hardware-based firewall. This is analogous to the direct memory access (DMA) controller on most modern 
microprocessors. 


\section{RISC-V processor}


\section{Firewall}

Usually, the first line of defence against bad actors, it is a vital component in a computer network and can become vastly complex. There are several
types of Firewalls such as packet filters (pf), stateful packet firewalls and application firewalls \cite{FirewallsBook}. Firewalls can also perform 
other tasks and employ other techniques to secure a network, however, in this project the most basic pf-style firewall will be implemented. 
Packet filters are considered as stateless and traditionally only filter on the fields in the headers in the network (layer 2) and transport 
(layer 3) layers \cite{FirewallsBook}. Such fields include IP addresses, port numbers and protocol type.

More advanced firewalls can perform deep packet inspection and explore the contents of the higher layers to better evaluate a packets true intention. 
While there is provision to add this functionality on an FPGA based firewall, this will not be explored in this project due to its significant increase 
in complexity. 