
% ***************************************************
% Example of an internal chapter
% ***************************************************
%This is an internal chapter of the thesis.
%If you have a long title, you can supply an abbreviated version to print in the Table of Contents using the optional argument to the \chapter command.
\chapter[Literature review]{Literature review }
\label{Chap:label}	%CREATE YOUR OWN LABEL.
\pagestyle{headings}



Some of the concepts behind the proposed project, such as an Ethernet MAC or RISC-V processor are not new, there is a variety of previous works in these areas. This part of the proposal will explore the prior work related to the project. 



\section{RISC-V processor}
Due to the open nature of the RISC-V architecture, many designers have made their own System on Chips (SoC) and created their own implementations. RISC-V 
themselves have published a list\footnote[1]{See: https://github.com/riscv/riscv-isa-manual/blob/master/marchid.md} of different RISC-V implementations 
that have a unique architecture ID. The majority of these are either written for application specific integrated circuits (ASICs), written in a hardware 
description language other than VHDL or has poor documentation with the exception of the  \textit{NEORV32 RISC-V} softcore processor. This SoC written 
in vendor agnostic VHDL and importantly has a considerable amount of documentation. 

Being a softcore processor, there is control over which modules are implemented and are not. Some basic features of the \textit{NEORV32 RISC-V} include 
UART, SPI, and GPIO interfaces \cite{neorv32Datasheet}. The datasheet \cite{neorv32Datasheet} also mentions that it supports a \textit{'Wishbone b4'} 
external bus interface. 


A Wishbone B4 (refered to as just 'wishbone') interconnection is designed specifically to connect peices of hardware together on a System-on-Chip (SoC) \cite{WishboneSpec}. In the NEORV32, 
It allows for external hardware modules to be memory mapped into the the 32bit address space on the processor \cite{neorv32Datasheet}.




\section{Ethernet MAC}

First introduced in 1983 \cite{IEEE802.3-2012}, the IEEE 802.3 standardardised a technology, Ethernet, to interconnect devices. There have been 
many attempts at creating hardware for Ethernet MACs. 





The decision in creating a custom MAC might be considered as interesting given the range of pre-existing Intellectual 
Property (IP) cores for Ethernet on FPGAs. The issue with the pre-existing solutions only have a single output to connect to something like a softcore
processor. To create a firewall, the network traffic would need to pass through the processor. To decrease latency, a second interface can be added 
to the MAC to allow traffic to flow through a hardware-based firewall. This is analogous to the direct memory access (DMA) controller on most modern 
microprocessors. 



\section{Packet Filter Firewall}

Usually, the first line of defence against bad actors, it is a vital component in a computer network and can become vastly complex. There are several
types of Firewalls such as packet filters (pf), stateful packet firewalls and application firewalls \cite{FirewallsBook}. Firewalls can also perform 
other tasks and employ other techniques to secure a network, however, in this project the most basic pf-style firewall will be implemented. 
Packet filters are considered as stateless and traditionally only filter on the fields in the headers in the network (layer 2) and transport 
(layer 3) layers \cite{FirewallsBook}. Such fields include IP addresses, port numbers and protocol type.

More advanced firewalls can perform deep packet inspection and explore the contents of the higher layers to better evaluate a packets true intention. 
While there is provision to add this functionality on an FPGA based firewall, this will not be explored in this project due to its significant increase 
in complexity. 


\section{RISC-V processor}






The IEEE 802.3 standard\cite{IEEE802.3-2012}, more commonly known by the name of 'Ethernet' defines the \textit{'Medium Access Control'} (MAC) 
protocol amongst other things for two or more devices to communicate over a network. This standard is just one part in the layered network 
models such as the OSI model and TCP/IP models. 


RISC-V, a reduced instruction set computer architecture, is an open and royalty free instruction set architecture (ISA). As a result, a plethora 
of soft-core processors have been made. The specific core proposed in this project is the 'NEORV32` RISC-V processor. It's a highly configurable 
microcontroller like system on chip (SoC) written purely in VHDL.

