\chapter[Introduction]{Introduction}
\label{Chap:Intro}

% ***************************************************
% Introduction
% ***************************************************


\section{Motivation}


In a technology era of increasing numbers of cyber attacks and record number of connected devices, ensuring these devices operate safely and securely is 
paramount. The Australian Cyber Security Center (ACSC) received in 
excess of 76,000 cybercrime reports and growing in the 2021-22 financial year \cite{acsc_2022}. The growing trend of Internet of Things (IoT) will provide 
more opportunity for black hats (malicious attackers). IHS Markit estimates 125 billion IoT devices will be connected by 2030 \cite{IHS_iot}.
This proliferation of IoT devices necessitates robust and adaptable security measures to counter the evolving threats posed by malicious actors. 

To manage the surge of IoT devices, a shift to edge computing has emerged in favour of the traditionally more centralised cloud computing 
architectures. Edge computing as \cite{EdgeComputing} puts it, is the paradigm which involves the computation and analysis of data 
at the \textit{edge} of the network to be as close as possible to the source of the data. This has many advantages including: lower latency, lower bandwidth requirements,
enhanced availability, energy efficiency, improved security and privacy \cite{EdgeComputing}. Consequently, smaller and more efficient computers can be deployed 
at the edge/perimeter of these networks \cite{EdgeComputingPerspectives}. 

Just like any other computer connected to the broader network, edge networks 
must also be safeguarded from malicious bad actors. Field programmable gate arrays (FPGAs) offer the flexibility of custom hardware that can be designed to incentivise low latency, high throughput yet efficient wire-speed firewalls. While current hardware firewalls exist in todays markets, they often come at a cost and high power usage rendering them unsuitable in edge networks. To address this, this thesis proposal attempts to design a FPGA firewall that fulfils these criteria.

