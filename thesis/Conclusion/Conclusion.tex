\chapter[Conclusion]{Conclusion}
\label{Chap:Conclusion}


\section{Limitations}
While this thesis explored the design and implementation of a hardware packet filter, Ethernet MAC with a RISC-V softcore processor, there are some limitations to the design and the research conducted. These can be summarised in the following points:


\begin{itemize}
    \item The current system only supports filtering network packets in one direction as it assumes all packets leaving the device is safe. 
    \item The design only considers IPv4 packets without IEEE 802.1Q VLAN tagging.
    \item The transmit logic is not optimised for resource usage and can be improved.
    \item Only one interface is supported due to the bandwidth limitations of the PMOD ports on the Nexys A7 board.
    \item Only HTTP and other unencrypted protocols are supported.
    \item The current design is bottlenecked from the processor - NEORV32
\end{itemize}


In addition, only one interface is supported due to the Nexys A7 board only consisting of one ethernet PHY and the additional PMOD ports are not suitable for Ethernet. This is because the PMOD ports are only rated for a 25Mhz bandwidth, while the RMII signals are 50Mhz. As such, signal integrity issues arose (see appendix \ref{app:eye_diagrams} for eye diagram) and restricted the use to just one interface - the onboard PHY. A new development board with two PHYs would be needed.











\section{Sustainability}




Ethernet standards have been in place since 1985 \cite{IEEE802.3-2012} and the IP, TCP, UDP and other layer 4 protocols have been in service for decades and have not changed. Typically when new features are added in these areas, they are added ontop of and not modify the packet structure itself. Such event was in 1998 with the introduction of IPv6 \cite{rfc2460}.

While malicious bad actors come up with new and innovative ways to breach a system, the core of this packet filter is solid and fundamentally should not need to change in future. Instead other additions and modification should be made to the system architecture to improve security, including but not limited to, HTTPS, public key cryptography and deep packet inspection.

Using the neorv32 while is feature-rich, as it's in its developement process and new features are continually getting added and existing features moved around, it doesn't make for the best choice for ease of updating, but it does get regularly updated so if there was a security vulnerability, odds are that it will get patched.

Similarly, the FreeRTOS-Plus-TCP is a first-party library for FreeRTOS and is actively maintained. It is also feature-rich and is continually getting updated, however, it's documentation and community support is more limited than that of LwIP.

Perhaps a larger issue with sustainability is the lifecyle of a Vue.js webapp. When updates are needed, potentially new versions of the toolchain may be needed thus contributing to longer development times. 

Choice of riscv makes this project easy to commercialise since there is no licences needed for the risc-v cores are royalty free and the specific variant, neorv32 is open sourced under the BSD-3-Clause license. \footnote[1]{See: https://github.com/stnolting/neorv32/blob/main/LICENSE}

This design was deployed on a Xilinx Artix 7 FPGA board with the LAN8720A PHY. As the LAN8720A chip uses the standardised RMII interface, there should be no issues porting the project to another FPGA or FPGA board so long as there is sufficient LUTs and BRAM available on the FPGA and uses an RMII phy. The hardware could be updated to RGMII or XGMII without too many issues with the main difference being in the input and output FIFO being able to support the different clock rates and bit widths.  













\section{Recommendations and future work}

In light of the findings in this thesis, several key recommendations can be made for future work in the area of embedded system SoC design which features Ethernet connectivity.

The primary recommendation is to incorporate dedicated hardware packet filtering into SoC designs. As demonstrated in this thesis, the resource utilisation for the packet filtering logic is minimal (571 slice LUTs and 1145 slice registers) and can be easily integrated into the design with minimal impact to cost. Superior latency, throughput and power consumption metrics are only some of the benefits presented in this thesis over the conventional software based packet filters. Additionally, the potential resilience against potential security vulnerabilities is another key advantage of this approach.

While the NEORV32 is a solid general-purpose softcore processor, it seemed to bottleneck the design and witheld the design from achieving better performance. The CVITEK CV1800B, used in the MilkV-Duo and compared in this thesis, is a powerful SoC with an abundance of resources including a hardware MAC and PHY, but falls short of including a hardware packet filter. An ideal choice would be to have the performance of the CVITEK CV1800B with the hardware packet filtering capabilities of the design presented in this thesis. This would give the best of both worlds, performance and security.

Alternatively, research into using hybrid SoC FPGAs such as the Xilinx Zynq lineup which include an FPGA and a hardcore processor connected over a high speed fabric could be a good avenue. This provides the flexibility of an FPGA with the performance of a hardcore processor, ideal for small scale designs that would otherwise be too expensive for custom silicon.

Leveraging single page application frameworks such as Vue.js for use in embedded systems is another recommendation resulting from the work done in this thesis. Light-weight applications and low power devices can benefit greatly from the use of such frameworks as fewer network traffic is required due to client-side routing and static web content. In combination with a lightweight API, dynamic data can be obtained with minimal network traffic, making the user experience seamless and responsive.

In addition to these recommendations for future designs, the work in this thesis can be extended in the following areas:

\begin{itemize}
    \item Redesign of the transmit logic to consume less resources while not losing on speed/performance, 
    \item Add a second Ethernet interface to filter traffic for other devices on the network, 
    \item Utilise the DDR2 RAM to free up BRAM in the FPGA, 
    \item Implement public key cryptography for HTTPS,
    \item Support faster media interfaces, eg RGMII for 1Gbit/s or XGMII for 10Gbit/s,
    \item Use a different bus interconnect for the processor, eg AXI4, and
    \item Look into using LwIP over FreeRTOS-Plus-TCP.
\end{itemize}




\section{Summary}

This thesis explored the design and implementation of a hardware packet filter, Ethernet MAC with a RISC-V softcore processor. The design was implemented on a Xilinx Artix 7 FPGA board with the LAN8720A PHY. The design was able to achieve a throughput of 94Mbit/s with a latency of 1.04ms. The design was also able to achieve a power consumption of 128mA at 94Mbit/s. 