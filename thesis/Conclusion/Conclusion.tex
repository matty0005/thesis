% ***************************************************
% Conclusion
% ***************************************************
\chapter[Conclusion]{Conclusion}
\label{Chap:Conclusion}

% ********* Enter your text below this line: ********
Conclude your thesis.

% ***************************************************


\subsection{Improvements}

\begin{itemize}
    \item Bidirectional filtering - currently only doing incomming filtering
    \item Redesign of the transmit logic to consume less resources while not losing on speed/performance - can reduce number of buffers. 
    \item Add a second interface - would need a new pcb and would need onboard switch if webserver was also wanted.
    \item Utilise the DDR2 RAM onboard the Nexys a7 board for RAM to free up BRAM. 
    \item Use another FPGA board with only the essential peripherals (eg dont need USB, Audio, temperature sensor etc on Nexys board)
\end{itemize}


\subsection{Recommendations}

Add dedicated hardware packet filtering to SoC based computers - safer - faster, efficient and lightweight

Reduce number of transmit buffers needed - consolidate them.

Use an FPGA with more BRAM as this would help with filter sizes, buffers and also ram needed for the RTOS and webservers.

Consider Hybrid SoC FPGA such as the Xilinx Zynq lineup.

Hardware ethernet and packet filter reduces latency, but is bottlenecked by CPU.

Use a different bus which the processor can handle - wishbone is overkill ~2.5Gbit for 100Mbit connection.

If power efficiency is desired, consider something like the MilkV duo. 128mA at 94mbit/s

The NeoRV32 is a solid softcore processor, but it's size grows exponentially when you use the wishbone interface. It's feature set is growing and while good, is not a great idea in production devices. 


While Freertos plus TCP provides a good library, it's documentation and community support is lackluster in comparison to LwIP. But is relatively easy to use. 

Using Vue.js to write the website in is a good idea as it reduces the network traffic between the device as routing is handled on the client side. So for lightweight applications, it is ideal. No need to have a special webserver for it as the site is static. However a separate API is needed, but easy to implement. 
