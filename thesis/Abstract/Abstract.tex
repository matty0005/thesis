% ***************************************************
% Abstract
% ***************************************************
% TO PRODUCE A STAND-ALONE PDF OF YOUR ABSTRACT, uncomment this section and the \end{document} at the end of the file by removing the % from the start of each line.

%\documentclass[12pt, a4paper]{memoir}

%\input{LaTexPackages.tex}

%\begin{document}

%\begin{center}
	%\textbf{\large Your title goes here}

	%\textbf{Abstract}

	%Your Name, The University of Queensland, 20??
%\end{center}


This thesis presents the design and implementation of both a hardware Ethernet Media Access Control (MAC) and packet filter on a Xilinx Artix 7 100T FPGA, specifically with the Digilent Artix 7 FPGA development board which includes a Reduced Media-Independent Interface (RMII) physical (PHY) interface chip. The primary objective of this work was to implement a firewall to improve security in the embedded systems space and to then host a web server on an onboard RISC-V softcore for configuration. More specifically, a NEORV32 RISC-V System on Chip (SoC) was used to interface the hardware over a Wishbone bus with the software hosting the webserver with FreeRTOS utilising both the Freertos-Plus-TCP and FreeRTPS-Plus-FAT libraries. 

The wirespeed hardware five-tuple packet filter, analysing the destination IP, source IP, destination port, source port and protocol, showcased an added delay of just $4\mu s$ irrespective of packet lengths while potentially enhancing security over software based implementations. Many performance benchmarks were also conducted and concluded in a relative power draw of 0.51W including the microprocessor. In comparison other platforms such as the Nucleo-F767ZI, Raspberry Pi Pico with WIZ5500 and MilkV-Duo were evaluated for their performance and efficiency.  

In addition, the web server hosted a static single page application style website using Vue.js and Tailwindcss which was all stored on a microSD card and accessed over the SPI interface and using the FAT32 filesystem. UDP round trip times were also measured for all platforms resulting in an average delay of 1.45ms for the FPGA board which included an added 1ms delay. 

Although effective, the packet classifier lacks support for IPv6 and only is applied to incoming traffic, while the firmware forgoes support for HTTPS. Given the FPGA's resource consumption of 11,738 slice LUTs and 12,505 slice registers, potential optimisations are discussed to overcome these shortcomings. A recommendation for future designs includes incorporating the efficiency and performance of the MilkV Duo RISC-V (CVITEK CV1800B based) board with an integrated hardware packet filter for a fast and secure embedded system platform. 

%\end{document}