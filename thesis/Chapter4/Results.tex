
% ***************************************************
% Example of an internal chapter
% ***************************************************
%This is an internal chapter of the thesis.
%If you have a long title, you can supply an abbreviated version to print in the Table of Contents using the optional argument to the \chapter command.
\chapter[Results]{Results}
\label{Chap:label}	%CREATE YOUR OWN LABEL.
\pagestyle{headings}


To ensure that the firewall was performing properly, a few tests were conducted. In this chapter, the testing results for the performance and the resource utilisation among other features are discussed. 

\section{Modifications}
The design was changed from the 2 ethernet interfaces to a design with just a single ethernet interface. 

\section{Performance}

Talk about wishbone bus speed. 
Initial tests were also conducted at 100MHz, but due to timing issues, it was reverted to 50MHz.


To test the speed of the packet classifier, a Agilent MSO6054A MSO was used (4GSa/s). A pin was set to the output of the crs\_dv of the PHY and the crs\_dv after the packet filter. The time between the rising and falling edge of the output will be the delay added by the packet filter. 


As the shift register in the hardware has a length of 224 bits, at a clock frequency of 50Mhz, the added latency is $224 \times \frac{1}{50\times 10^6} = 4.48 \times 10^{-6} = 4.48uS$. 


\begin{figure}[h!]
    \centering
    \includegraphics[width=0.75\textwidth]{Images/scope_4.png}
    \caption[Added latency by packet filter waveform]{Added latency by packet filter waveform.}
    \label{fig:packet_classifier_architecture}
\end{figure}

There is two pulses in this graph. This is due to when the crs\_dv line gets disasserted at the end of the packet. From this the time taken to receive the packet can be found to be 8.94uS. This is just an observation and is proportional to the packet size.



The same setup was used to test the preexisting solution, except this time GPIO pins were set high and low. It is assumed that the latency of setting the pin high cancels out with the latency of setting the pin low. 

\subsection{Limitations}
\subsubsection{PMOD Interface}

There are 5 PMOD connectors on the development board. Initally, one of these would be used for a second Ethernet PHY, but due to bandwidth limitations of the interface, the design had to be altered. The recommended bandwidth of these ports are 25MHz while the Ethernet RMII PHY would have been using 50Mhz signals over the interface. As such, signal integrity issues arose (see figure \ref{fig:eye_diagram}) and restricted the use to just one interface - the onboard PHY. A new development board with two PHYs would be needed.

\begin{figure}[h]
    \centering
    \includegraphics[width=0.65\textwidth]{Images/EyeDiagramTX.png}
    \caption[Eye diagram of TXD through PMOD interface]{Eye diagram of TXD through PMOD interface.}
    \label{fig:eye_diagram}
\end{figure}


\subsection{Testing setup}

\subsection{Results}



\section{Ultilisation}

Could be a big section where i take a look at the design against the number of rules - show power usage and gate consumption.


\section{Timing constraints}
\label{sec:timing_constraints}





Testing with WIZ5500 Pico, avg udp rtt was 1.88ms from 1000 tests with a payload of 7 bytes.
Testing with WIZ5500 Pico, avg udp rtt was 2.07ms from 1000 tests with a payload of 256 bytes.

Testing with FPGA board, avg udp rtt was 1.48ms from 1000 tests with a payload of 7 bytes. Dropped 39 packets
Testing with FPGA board, avg udp rtt was 1.86ms from 1000 tests with a payload of 256 bytes. Dropped 42 packets

observations, the WIZ5500 chip was getting excessively hot..

Maybe do a power and thermal camera comparison.


A thermal camera was used to record the temperatures periodically. At an ambient room temperature of $24.8\degree C$ throughout the test, after 5mins the WIZ5500 ethernet chip heated to $58.0\degree C$ and RP2040 was at $46.6\degree C$. While the FPGA was at $38.0 \degree C$. This is a bit of an unfair comparison as the physical size of the FPGA is much larger than the WIZ5500. After two hours of constant UDP ping requests to both devices, figure \ref{fig:thermal_2hr} shows the FPGA board and WIZ5500 board's temperature gradient. The FPGA plateaued to a maximum of $40.4 \degree C$ while the WIZ5500 was $1.c \degree C$ cooler at, $56.8 \degree C$. This could be due to accuracy of the measurements, in addition to not getting aiming the thermal camera in the hottest part. The RP2040 chip however was measured to be $53.2 \degree C$. Some additional thermal images can be found in the appendix.


\begin{figure}[h]
    \centering
    \includegraphics[width=0.5\textwidth]{Images/flir_2hs.jpg}
    \caption[Thermal image of FPGA board and WIZ5500 after two hours]{Thermal image of FPGA board and WIZ5500 after two hours.}
    \label{fig:thermal_2hr}
\end{figure}
